%! Author = Raffaele Talente
%! Date = 08/03/2023

% Preamble
\documentclass[12pt]{article}

% Packages
\usepackage{amsmath}
\usepackage{extsizes}
\usepackage{geometry}
\usepackage{titling}
\usepackage{textcomp}
\usepackage{stmaryrd}
% Document
\title{\vspace{-1.0cm}Recurrences}
\date{}
\newgeometry{vmargin={15mm}, hmargin={15mm,20mm}}

% Document
\begin{document}
\maketitle
\vspace{-2.0cm}
A \textit{recurrence} is an equation or inequality that describes a function in terms of its value on smaller inputs. \newline
\hspace*{5mm}Solving a recurrence means obtaining an asymptotic bound on the solution.
There are 3 methods for solving recurrences:
\begin{itemize}
    \item The \textbf{substitution method} where we guess a bound and then prove or disprove it via mathematical induction.
    \item The \textbf{recursion-tree method} which converts the recurrence into a tree whose nodes
          represent the costs incurred at various levels of the recursion.
    \item The \textbf{master theorem method} which provides bounds for recurrences of the form \newline \newline
    $T(n) = a T(\frac{n}{b}) + f(n)$ \newline \newline
          where $a \geq 1, b > 1$ and $f(n)$ is a given function. \newline
          A recurrence of this form characterizes a divide-and-conquer algorithm that creates $a$ subproblems,
          each of which is $\frac{1}{b}$ the size of the original problem, and in which
          the divide and combine steps take $f(n)$ time.
\end{itemize}
\vspace*{10mm}
\begin{large}
    \textbf{The master theorem method}
\end{large} \newline \newline
Let $a \geq 1$ and $b > 1$ be constants, let $f(n)$ be a function and let $T(n)$ be defined on the nonnegative integers
by the recurrence \newline \newline
$T(n) = a T(\frac{n}{b}) + f(n)$. \newline \newline
Then $T(n)$ has the following asymptotic bounds:
\begin{enumerate}
    \item If $f(n) = O(n^{\log_ba - \epsilon}))$ for some constants $\epsilon > 0$, then $T(n) = \Theta(n^{\log_ba}))$
    \item If $f(n) = \Theta(n^{\log_ba}))$, then $T(n) = \Theta(n^{\log_ba}\lg n)$
    \item If $f(n) = \Omega(n^{\log_ba + \epsilon}))$ for some constants $\epsilon > 0$ and if $af(\frac{n}{b}) \leq cf(n)$
          for some constant $c < 1$ and all sufficiently large $n$, then $T(n) = \Theta(f(n))$.
\end{enumerate}

\end{document}