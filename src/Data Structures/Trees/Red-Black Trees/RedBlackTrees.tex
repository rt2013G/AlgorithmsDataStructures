%! Author = Raffaele Talente
%! Date = 24/03/2023

% Preamble
\documentclass[12pt]{article}

% Packages
\usepackage{amsmath}
\usepackage{extsizes}
\usepackage{geometry}
\usepackage{titling}
\usepackage{textcomp}
\usepackage{natbib}
\usepackage{xcolor}
% Document
\title{\vspace{-1.0cm} Red-Black Trees}
\date{}
\newgeometry{vmargin={15mm}, hmargin={15mm,20mm}}

% Document
\begin{document}
    \maketitle
    \vspace{-2.0cm}
Red-Black Trees (\textit{RBT}) are Binary Search Trees \textit({BST}) in which every node contains one additional information: its color,
which can either be red or black.
Compared to BSTs, RBTs guarantee that basic dynamic-set operations take at most $O(log_2 \ n)$.
In a RBT, keys are stored in internal nodes, and leaves as well as the parent of the root are replaced
by a sentinel node \textbf{NIL}.
RBTs are BSTs that satisfy the following Red-Black properties:
\begin{enumerate}
    \item Each node is either \textit{red} or \textit{black}.
    \item The root is \textit{black}.
    \item All leaves (NIL) are \textit{black}.
    \item If a node is \textit{red}, then both its children are \textit{black}.
    \item All simple paths from a node to a leaf contain the same number of \textit{black} nodes.
\end{enumerate}
The number of nodes on any simple path from a node \textit{x} (excluded) down to a leaf is called
\textbf{black-height} of x, or \textbf{b-height}, denoted by $bh(x)$. \newline
The black-height of the tree is the black-height of the root. \newline \newline
\textbf{LEMMA}: A RBT with $n$ internal nodes has a height at most $2 \ log_2(n + 1)$ \newline \newline
    \textbf{Proof}: \newline
We first prove by induction that any subtree with root $x$ has at least $2^{bh(x)} - 1$ internal nodes.
If the height is 0, then x is a leaf, therefore it has $2^0 - 1 = 0$ internal nodes. \newline
Any internal node x has 2 children, both of which have a black-height of either $bh(x)$ or $bh(x) - 1$
depending if its red or black, respectively. \newline
Therefore, each child has at least $2^{bh(x) - 1} - 1$ internal nodes, thus the subtree rooted at x
as at least: \newline
    $((2^{bh(x) - 1} - 1) + (2^{bh(x) - 1} - 1) + 1)$ \textit{(both children + the root itself)} = \newline \newline
$2^1 \ 2^{bh(x) - 1} - 2 + 1$ = $2^{bh(x) - 1}- 1$. \newline \newline
According to property \textit{(4)}, at least half of the nodes from the root down to a leaf must be black,
therefore, the b-height of the root must be at least $2^{\frac{h}{2}} - 1$: \newline
$n \geq 2^{\frac{h}{2}} - 1$ \newline \newline
$n + 1 \geq 2^{\frac{h}{2}}$ \newline \newline
$log_2(n + 1) \geq \frac{h}{2}$ \newline \newline
\textbf{$h \leq 2 log_2(n + 1)$} \newline \newline \newline
Insertion and Deletion in a RBT may lead to violations of the Red-Black properties. \newline
In order to restore such properties, we need to
    \begin{enumerate}
        \item Change the pointer structure of the tree
        \item Change the color of some nodes
    \end{enumerate}
We change the pointer structure through \textit{rotations}, which may either be \textit{left rotations}
    or \textit{right rotations}. \newline
In the left rotation of a Node \textit{x}, we assume its right child \textit{y} to be $\neq NIL$, \textit{x} then
    becomes \textit{y}'s left child, and \textit{y}'s left child becomes \textit{x}'s right child. \newline
    The \textit{right rotation} is completely symmetric. \newline
Rotations run in $O(1)$ time, because only pointers are changed by them while other attributes in a node
    remain the same. \newline \newline \newline
\textbf{Insertions} \newline \newline
We can insert a node in a RBT in $O(\log(n))$ time.
To do so, we first insert the new node \textit{Z} in the RBT as if it were a normal BST, we then color \textit{Z} red.
Since this insertion may violate the \textit{Red-Black properties} we call an auxiliary procedure \textit{insertFixup}
to recolor nodes and performs rotations. \newline
There are four cases to consider: since we added a \textit{red} node, the only properties that the insertion may violate
are \textit{(2)} and \textit{(4)}.
\begin{enumerate}
    \item If the tree violates property \textit{(2)}, then \textit{Z} is the root, which is the only internal
    node of the tree.
    Since its parent and children are sentinels, property \textit{(4)} cannot be violated, therefore, we simply recolor
    \textit{Z} as black.
    \item If the \textit{Z}'s uncle is red, then the tree violates property \textit{(4)}, therefore, we recolor
    \textit{Z}'s uncle, parent and grandparent.
    \item If \textit{Z}'s uncle is black, and either \textit{Z}'s parent is a right child and \textit{Z} is a left child
    OR \textit{Z}'s parent is a left child and \textit{Z} is a right child, we rotate \textit{Z}'s parent in
    the opposite direction of \textit{Z}.
    \item If \textit{Z}'s uncle is black, and both \textit{Z}'s parent and \textit{Z} are right children OR
    both \textit{Z}'s parent and \textit{Z} are left children, we rotate \textit{Z}'s grandparent in the opposite
    direction of \textit{Z}, we then recolor the original parent and grandparent of \textit{Z}.
\end{enumerate}
    Insertion in a RBT has a time complexity of $O(\log(n))$. \newline
    The basic inserting operation takes $O(\log(n))$, we then perform $\log(n)$ recolors and rotations,
    which take $O(1)$ time. \newline \newline \newline
\textbf{Deletions} \newline \newline
Just like insertions, deletions in a RBT may lead to a violation of the Red-Black properties. \newline
To delete a node \textit{Z}, we first delete it like we would in a normal BST, with a slightly different
    \textit{transplant} procedure and by keeping track of a node \textit{Y} that may violate the properties, then
    call an additional procedure \textit{deleteFixup} which recolors nodes and performs rotations. \newline
If \textit{Z}'s original color was red, then red-black properties still hold:
    \begin{enumerate}
        \item No black-heights have changed.
        \item No red nodes have been made adjacent, because \textit{Y} takes \textit{Z}'s place and its color.
        \item Since \textit{Y} could not have been the root, the root remains black.
    \end{enumerate}
If \textit{Z}'s original color was black, then removing or moving \textit{Y} may violate the Red-Black properties,
    as any path containing \textit{Y} may have a \textit{"missing black"}. \newline
    Therefore, we add an \textit{"extra black"} to \textit{X} and try to move it up the tree until either
    \textit{X} is both red and black - at which point we simply color it black - or \textit{X} is the root, at which
    point we can remove the extra black. \newline
    We have to handle 4 cases (and their symmetrical variants), depending on the node \textit{X}
    and its sibling \textit{W}:
    \begin{enumerate}
        \item If \textit{W} is red, both of \textit{W}'s children are black, we swap the colors of \textit{W} and
        and \textit{X}'s parent, and perform a rotation on \textit{X}'s parent on the same side of \textit{X}. \newline
        \textit{X}'s sibling \textit{W} is now black, therefore, we have moved to one of the other 3 cases.
        \item If \textit{W} is black and both of its children are black, we change \textit{W}'s color to red and repeat
        the loop on \textit{X}'s parent.
        \item If \textit{W} is black and its right child is black (and \textit{W}'s left child is red),
        we recolor \textit{W} and \textit{W}'s left child, and then perform a right rotation on \textit{W}.
        \textit{X}'s new sibling is black with a red child, therefore, we are now in the \textit{4th} case.
        \item If \textit{W} is black and its right child is red, we transfer to \textit{W} the color of its parent and
        then change the color of \textit{W}'s parent and \textit{W}'s right child to black. \newline
        We then perform a left rotation on \textit{X}'s parent and set \textit{X} as the root
        (causing the loop to terminate).
    \end{enumerate}
    The time complexity of the delete operation is $O(\log(n))$ without the \textit{deleteFixup} procedure, because
    the height of a tree of $n$ nodes is at most $\log(n)$. \newline
    Within the \textit{deleteFixup} procedure, \textit{case 1, 3 and 4} only perform a constant number of color changes
    and rotations. \newline
    \textit{Case 2} is the only case in which the while loop can be repeated, and then the pointer \textit{X} moves up
    the three at most $\log(n)$ times. \newline
    Therefore, the overall time complexity of the delete operation is $O(\log(n))$.

\end{document}
